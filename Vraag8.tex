De verschillende $\alpha$ waarden hebben een invloed op de ligging van de eigenwaarden alsook het conditiegetal $\kappa(A)$ van A. Voor $\alpha = 1$ is het conditiegetal $\kappa(A) = 9.4108*10^3$, voor $\alpha = 5$ is $\kappa(A) = 46.6201$, voor $\alpha = 10$ is $\kappa = 13.3134$ en tot slot voor $\alpha = 100$ is $\kappa(A) = 1.5601$.\\[12pt]

Wanneer we nu de eigenwaarden gaan bekijken zien we dat de eigenwaarden bij \ref{alpha 1} het meest verspreid liggen en het meest naar het centrum, hoe hoger de $\alpha$ waarde hoe verder de meeste eigenwaarden (behalve het nulpunt) zich naar de rand van de eenheidscirkel begeven en dichter bij elkaar komen te liggen. Dit komt ook overeen met de eigenschap: $\|r_{n}\| \leq inf\|p(A)\| \leq \kappa_{2}(V) inf (max |p(\lambda)|\|r_{0}\|),\forall p\in P_{n}, \forall\lambda\in \sigma(A)$.\\
 Deze stelt immers dat snelle convergentie bereikt wordt voor eigenwaarde van A die geclusterd zitten, weg van de oorsprong en met A gelijkend op een normaal matrix ($A*A^{*} = A^{*}*A$).

\begin{figure}[!tbp]
  \centering
  \subfloat[alpha 1]{\includegraphics[width=0.5\textwidth]{Tekeningen/GMRES_alpha1_eigenv}\label{alpha 1}}
  \hfill
  \subfloat[alpha 5]{\includegraphics[width=0.5\textwidth]{Tekeningen/GMRES_alpha5_eigenv}\label{alpha 5}}
  \caption{Links eigenwaarden voor $\alpha = 1$, rechts voor $\alpha = 5$}
\end{figure}

\begin{figure}[!tbp]
  \centering
  \subfloat[alpha 10]{\includegraphics[width=0.5\textwidth]{Tekeningen/GMRES_alpha10_eigenv}\label{alpha 10}}
  \hfill
  \subfloat[alpha 100]{\includegraphics[width=0.5\textwidth]{Tekeningen/GMRES_alpha100_eigenv}\label{alpha 100}}
  \caption{Links eigenwaarden voor $\alpha = 10$, rechts voor $\alpha = 100$}
\end{figure}

\begin{figure}[!tbp]
  \centering
  \subfloat[fout op $\|r_{n}\|$]{\includegraphics[width=0.5\textwidth]{Tekeningen/GMRES_alpha1_r}\label{r_1}}
  \hfill
  \subfloat[fout op exacte oplossing]{\includegraphics[width=0.5\textwidth]{Tekeningen/GMRES_alpha1_x-y}\label{x-y_1}}
  \caption{Resp. fouten voor $\alpha = 1$.}
\end{figure}

\begin{figure}[!tbp]
  \centering
  \subfloat[fout op $\|r_{n}\|$]{\includegraphics[width=0.5\textwidth]{Tekeningen/GMRES_alpha5_r}\label{r_5}}
  \hfill
  \subfloat[fout op exacte oplossing]{\includegraphics[width=0.5\textwidth]{Tekeningen/GMRES_alpha5_x-y}\label{x-y_5}}
  \caption{Resp. fouten voor $\alpha = 5$.}
\end{figure}

\begin{figure}[!tbp]
  \centering
  \subfloat[fout op $\|r_{n}\|$]{\includegraphics[width=0.5\textwidth]{Tekeningen/GMRES_alpha10_r}\label{r_10}}
  \hfill
  \subfloat[fout op exacte oplossing]{\includegraphics[width=0.5\textwidth]{Tekeningen/GMRES_alpha10_x-y}\label{x-y_10}}
  \caption{Resp. fouten voor $\alpha = 10$.}
\end{figure}

\begin{figure}[!tbp]
  \centering
  \subfloat[fout op $\|r_{n}\|$]{\includegraphics[width=0.5\textwidth]{Tekeningen/GMRES_alpha100_r}\label{r_100}}
  \hfill
  \subfloat[fout op exacte oplossing]{\includegraphics[width=0.5\textwidth]{Tekeningen/GMRES_alpha100_x-y}\label{x-y_100}}
  \caption{Resp. fouten voor $\alpha = 100$.}
\end{figure}


