De definitie van het Rayleigh quoti\"ent is:\\

$r(x) = \dfrac{x^{T}Ax}{x^{T}x}$

Bij eigenvectoren volgt hier een makkelijke omzetting:\\

\begin{equation}
\begin{gathered}
r(x) = \dfrac{v^{T}Av}{v^{T}v} = \dfrac{v^{T}\lambda v}{v^{T}v} = \lambda
\end{gathered}
\end{equation}

Volgens formule 27.2 uit Trefethen en Bau:\\

\begin{equation}
\begin{gathered}
\nabla r(x) = \dfrac{2}{x^{T}x}(Ax - r(x)x)
\end{gathered}
\end{equation}

Voor eigenvectoren wordt de term (Ax - r(x)x) = 0. Minima en maxima van het Rayleigh quoti\"ent liggen dus bij eigenvectoren. Uit (1) volgt dan dat de waarde hier gelijk is aan $\lambda$. Het maximum wordt dan bereikt bij $v_{max}$ en een minimum bij $v_{min}$. De waarden zijn daar $\lambda_{max}$ en $\lambda_{min}$.\\

Om het bewijs af te ronden is het nog nodig om te weten dat de eigenvectoren van A hier een orthonormale basis vormen voor de ruimte ${\rm I\!R}^{n}$. Alle vectoren kunnen dus voorgesteld worden als een lineaire combinatie van de eigenvectoren van A. Wanneer we dit combineren met de definitie van het Rayleigh quoti\"ent, vinden we:\\

\begin{equation}
\begin{gathered}
r(x) = \dfrac{\sum_{i=1}^{n}(\alpha_{i}v_{i})^{T}A\sum_{i=1}^{n}(\alpha_{i}v_{i})}{\sum_{i=1}^{n}(\alpha_{i}^{2}\|v\|^{2})} = \dfrac{\sum_{i=1}^{n}(\alpha_{i}^{2}\lambda_{i})}{\sum_{i=1}^{n}\alpha_{i}^{2}}
\end{gathered}
\end{equation}

De minimale waarde die kan bereikt worden is dus $\lambda_{min}$ en het maximum is $\lambda_{max}$. Uit (3) volgt ook meteen het bewijs voor het tweede deel. Elke waarde binnen het interval kan geschreven worden als een lineaire combinatie van eigenwaarden. Met deze co\"effici\"enten kan men dan de co\"effici\"enten bepalen voor de vector x als lineaire combinatie van eigenvectoren.\\