Wanneer we de tijdsduur berekenen van de gebruikte methode zien we dat het voor kleine matrices een klein verschil geeft, maar voor grote matrices het een grootteorde verschil is.\\[12pt]

\centering
\begin{tabular}{|r|c|c|c|}
\hline
Grootte van A & 10x10 & 100x100 & 1000x1000\\ \hline
Expliciet & 0.0022s & 0.0165s & 48.8148s\\ \hline
Impliciet & 0.0020s & 0.0082s & 4.3914s\\ \hline
\end{tabular}
\centering
\captionof{table}{Tijdsduur van Householder.}
\label{Tijd Householder}

\begin{flushleft}
Wanneer de fout (zowel op x als op het residu) nader bekeken wordt, blijkt deze nagenoeg hetzelfde te zijn. Het verschil in beide algoritmes bevindt zich dus in de tijdsduur, de impliciete berekening is dus de snelste. De resultaten van de fout zijn te vinden in bijlage 1.\\[12pt]
\end{flushleft}
